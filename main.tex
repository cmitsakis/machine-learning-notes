\documentclass{report}

\usepackage{amsmath}
\usepackage{bookmark}
\usepackage{geometry}
\geometry{
        paper=a4paper,
        top=2cm,
        bottom=2cm,
        left=1.5cm,
        right=1.5cm,
        headheight=0.75cm,
        footskip=1.5cm,
        headsep=0.75cm,
}
\usepackage{hyperref}
\usepackage{mdframed}
\usepackage[normalem]{ulem}
\usepackage{parskip}
\usepackage{unicode-math}

\title{Machine Learning Notes}


\begin{document}

\vspace*{2cm}
\begin{center}
	\Huge\textbf{Machine Learning Notes}\\
	\LARGE\textit{an introduction to machine learning}
\end{center}
\vspace*{2cm}

\textit{Machine Learning Notes} \url{https://github.com/cmitsakis/machine-learning-notes} \\
Copyright © 2024 by \textit{Charalampos Mitsakis} \url{https://github.com/cmitsakis} \\
is licensed under \textit{Creative Commons Attribution-ShareAlike 4.0 International (CC-BY-SA 4.0)}. \\
To view a copy of this license, visit \url{https://creativecommons.org/licenses/by-sa/4.0/}


\chapter{Probability Theory}

\paragraph*{Conditional probability}
\[P(A \mid B) = \frac{P(A \cap B)}{P(B)}\]

\paragraph*{Sum rule}
\[P(A)=\sum_n P(A \cap B_n)\]
or, alternatively
\[P(A)=\sum_n P(A \mid B_n) P(B_n)\]

\paragraph*{Product rule}
\[P(A, B) = P(A) P(B \mid A)\]

\paragraph*{Bayes theorem}
\[P(A \mid B) = \frac{P(B \mid A) P(A)}{P(B)}\]
\[P(A_i \mid B) = \frac{P(B \mid A_i) P(A_i)}{\sum\limits_j P(B \mid A_j) P(A_j)}\]

\paragraph*{Probability densities}
\[P[x \in (a, b)] = \int_a^b p(x)\ dx\]
cumulative distribution function:
\[P(z) = \int_{-\infty}^z p(x)\ dx\]
sum rule:
\[p(x) = \int p(x, y)\ dx\]
product rule:
\[p(x, y) = p(x) p(y \mid x)\]

\paragraph*{Expectation}
\[E[X] = \sum_{i=1}^\infty x_i p_i\]
\[E[g(X)] = \sum_{x} g(x) p(x)\]
\[E[X] = \int_{-\infty}^\infty x f(x)\ dx\]
\[E[g(X)] = \int_{-\infty}^\infty g(x) f(x)\ dx\]

\paragraph*{Variance and covariance}
\[\begin{split}
	\operatorname{Var}(X) &= E[(X - \mu)^2] \\
	&= E[X^2] - E[X]^2 \\
	&= \operatorname{Cov}(X, X)
\end{split}\]
\[\begin{split}
	\operatorname{cov}(X, Y) &= E{\left[(X - E[X])(Y - E[Y])\right]} \\
	&= E[X Y] - E[X] E[Y]
\end{split}\]
\paragraph*{Entropy}
\[H[X] = - \sum_x p(x) \log_2 p(x)\]


\chapter{Bayesian Decision Theory}

\[P(\omega_i \mid x) = \frac{P(x \mid \omega_i) P(\omega_i)}{P(x)} = \frac{P(x \mid \omega_i) P(\omega_i)}{\sum\limits_j P(x \mid \omega_j) P(\omega_j)}\]

\begin{mdframed}
	\textbf{videos}

	\href{https://www.youtube.com/watch?v=HZGCoVF3YvM}{Bayes theorem, the geometry of changing beliefs [3Blue1Brown]}

	\href{https://www.youtube.com/watch?v=4JscUHGWaB4}{Machine Learning: Bayes Decision Theory}
\end{mdframed}

\section{Minimum Error Classifier (MAP)}
A Bayesian classifier that minimizes the classification error probability.

For two classes:
\[P(\omega_1 \mid x) > P(\omega_2 \mid x) \Rightarrow x \in \omega_1\]
\[P(\omega_2 \mid x) > P(\omega_1 \mid x) \Rightarrow x \in \omega_2\]

equivalently:
\[P(x \mid \omega_1)P(\omega_1) > P(x \mid \omega_2)P(\omega_2) \Rightarrow x \in \omega_1\]
\[P(x \mid \omega_2)P(\omega_2) > P(x \mid \omega_1)P(\omega_1) \Rightarrow x \in \omega_2\]

For many classes:
\[P(\omega_i \mid x) > P(\omega_j \mid x),\ \forall j\ j \neq i \Rightarrow x \in \omega_i\]

\section{Minimimum Risk Classifier}
A Bayesian classifier that minimizes the total expected risk. It is useful for cases where some decisions are more important than others.

\subsection*{for two classes:}
\[L = \begin{bmatrix}
	\lambda_{11} & \lambda_{12} \\
	\lambda_{21} & \lambda_{22}
\end{bmatrix}\]
usually:
\[L = \begin{bmatrix}
	0 & \lambda_{12} \\
	\lambda_{21} & 0
\end{bmatrix}\]

\[l_1 = \lambda_{11} p(x \mid \omega_1)P(\omega_1) + \lambda_{21} p(x \mid \omega_2)P(\omega_2)\]
\[l_2 = \lambda_{12} p(x \mid \omega_1)P(\omega_1) + \lambda_{22} p(x \mid \omega_2)P(\omega_2)\]

\[x \in \omega_1 \text{ if } l_1 < l_2 \Leftrightarrow (\lambda_{21}-\lambda_{22}) p(x \mid \omega_2)P(\omega_2) < (\lambda_{12}-\lambda_{11}) p(x \mid \omega_1)P(\omega_1)\]

\[x \in \omega_1 (\omega_2) \text{ if } \frac{p(x \mid \omega_1)}{p(x \mid \omega_2)} > (<) \frac{P(\omega_2)}{P(\omega_1)}\frac{\lambda_{21}-\lambda_{22}}{\lambda_{12}-\lambda_{11}}\]

\section{Discriminant functions}
$g_i (\symbf{x}) \equiv f(P(\omega_i \mid \symbf{x}))$ where $f$ is a monotonically increasing function.

\[g_i(\symbf{x}) > g_j(\symbf{x}),\ \forall j\ j \neq i \Rightarrow x \in \omega_i\]

decision surfaces:
\[g_{ij}(\symbf{x}) \equiv g_i(\symbf{x}) - g_j(\symbf{x}) = 0,\ \forall i, j\ i \neq j\]

\subsection*{Bayesian classification for normal distributions}
\[g_i(\symbf{x}) = \ln p(\symbf{x} \mid \omega_i) + \ln P(\omega_i)\]
% Duda's book chapter 2.6 equation (47)
\[g_i(\symbf{x}) = -\frac{1}{2} (\symbf{x} - \symbf{\mu_i})^t \symbf{\Sigma_i}^{-1} (\symbf{x} - \symbf{\mu_i}) - \frac{d}{2}\ln2\pi - \frac{1}{2} ln |\symbf{\Sigma_i}| + ln P(\omega_i)\]
\paragraph*{Case 1: $\Sigma_i = \sigma^2 I$}
% Duda's book chapter 2.6.1 equation (51)
\[g_i(\symbf{x}) = \symbf{w_i}^t \symbf{x} + w_{i0}\]
where:
\[\symbf{w_i} = \frac{1}{\sigma^{2}} \symbf{\mu_i}\]
\[w_{i0} = -\frac{1}{2\sigma^2}\symbf{\mu_i}^t \symbf{\mu_i} + \ln P(\omega_i) \]

\paragraph*{Case 2: $\Sigma_i = \Sigma$}
% Duda's book chapter 2.6.2 equation (58)
\[g_i(\symbf{x}) = -\frac{1}{2} (\symbf{x} - \symbf{\mu_i})^t \symbf{\Sigma_i}^{-1} (\symbf{x} - \symbf{\mu_i}) - \frac{d}{2}\ln2\pi + ln P(\omega_i)\]
\[g_i(\symbf{x}) = \symbf{w_i}^t \symbf{x} + w_{i0}\]
where:
\[\symbf{w_i} = \symbf{\Sigma}^{-1} \symbf{\mu_i}\]
\[w_{i0} = -\frac{1}{2}\symbf{\mu_i}^t \symbf{\Sigma}^{-1} \symbf{\mu_i} + \ln P(\omega_i) \]

\paragraph*{Case 3: arbitrary $\Sigma_i$}
% Duda's book chapter 2.6.3 equation (64)
\[g_i(\symbf{x}) = \symbf{x}^t \symbf{W_i} \symbf{x} + \symbf{w_i}^t \symbf{x} + w_{i0}\]
where:
\[\symbf{W_i} = -\frac{1}{2}\symbf{\Sigma_i}^{-1}\]
\[\symbf{w_i} = \symbf{\Sigma_i}^{-1} \symbf{\mu_i}\]
\[w_{i0} = -\frac{1}{2}\symbf{\mu_i}^t \symbf{\Sigma_i}^{-1} \symbf{\mu_i} - \ln |\symbf{\Sigma_i}| + \ln P(\omega_i) \]


\chapter{Parameter Estimation}

\section{Maximum Likelihood Estimation}
\begin{mdframed}
	\textbf{videos}

	\href{https://www.youtube.com/watch?v=sguol03tfWo&list=PL5yR0euE9N2kGEf7gqysMFq0Spoq0evNf&index=2}{Machine Learning: Maximum Likelihood Estimation}

	\href{https://www.youtube.com/watch?v=XepXtl9YKwc}{Maximum Likelihood, clearly explained!!!}
\end{mdframed}

We have i.i.d. (independent and identically distributed) random samples $\symbf{X} = [\symbf{x_1}, \ldots, \symbf{x_n}]$ drawn from $p(\symbf{x} \mid \symbf{\theta})$ that follows a distribution that can described by a parameter vector $\symbf{\theta}$.
Find $\symbf{\theta}$ that makes the observed samples the most likely (which means it maximizes $p(X \mid \theta)$).

\[L_X(\theta) = p(X \mid \theta) = \prod_{k=1}^{n} p(x_k \mid \theta)\]

\[\hat \theta = \underset{\theta}{argmax}\ p(X \mid \theta)\]

\[\ell(\theta) = \ln L(\theta) = \sum_{k=1}^n \ln p(x_k \mid \theta)\]

Necessary condition for $\hat \theta$:
\[\frac{\partial p(X \mid \theta)}{\partial \theta} = 0\]
or:
\[\frac{\partial \ell(\theta)}{\partial \theta} = \sum_{k=1}^n \ln p(x_k \mid \theta) = \sum_{k=1}^n \frac{1}{p(x_k \mid \theta)} \frac{\partial p(x_k \mid \theta)}{\partial \theta} = 0\]

\end{document}
